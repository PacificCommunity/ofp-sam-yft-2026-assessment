\documentclass[aspectratio=169]{beamer}
\usepackage{spc}
\usepackage{textpos}
\begin{document}

\begin{frame}
  \title{\vspace{-5ex}\darkblue Yellowfin 2026 Assessment\\[2ex]
    \it\large\darkgray Initial Discussion Topics}
  \author{\vspace{-10ex}\darkgray\bf Arni Magnusson}
  \date{\darkgreen SPC Pre-PAW Meeting\\[0.5ex]
    13 November 2025}
  \titlepage
  \begin{textblock}{0}(10,0.5)
    \includegraphics[width=4cm]{fish_yellowfin}
  \end{textblock}
\end{frame}

% ______________________________________________________________________________

\begin{frame}{Overview}
  \vspace{2ex}
  \begin{itemize}
    \item[] {\bf\darkblue Innovations in 2023 Assessment}
    \comment{simplified regional structure, Lorenzen M,\\[1ex]
      \h{38.5ex} catch-conditioned, estimation uncertainty}\\[4ex]
    \item[] {\bf\darkblue Issues in 2023 Assessment}
    \comment{convergence/jittering, narrow confidence intervals,\\[1ex]
      \h{32.2ex} likelihood conflict between length and weight comps}\\[4ex]
    \item[] {\bf\darkblue New Issues}
    \comment{uncertainty about Indonesian catches}\\[4ex]
    \item[] {\bf\darkblue New Features in MULTIFAN-CL}
    \comment{length-based selectivity}\\[4ex]
    \item[] {\bf\darkblue Things to Report to PAW}
    \comment{initial findings related to the above}\\[4ex]
  \end{itemize}
\end{frame}

% ______________________________________________________________________________

\begin{frame}\Large
  \centering\darkgreen\bf Innovations in 2023 Assessment
\end{frame}

% ______________________________________________________________________________

\begin{frame}{Simplified Regional Structure}
  \includegraphics[width=0.47\textwidth]{previous_9_regions}
  \raisebox{13ex}{$\Rightarrow$}
  \includegraphics[width=0.47\textwidth]{previous_5_regions}
\end{frame}

% ______________________________________________________________________________

\begin{frame}{Lorenzen Natural Mortality}
  \vspace{4ex}
  \centering\includegraphics[height=0.6\textheight]{natmort}
\end{frame}

% ______________________________________________________________________________

\begin{frame}{Catch-Conditioned Method}\small
  MULTIFAN-CL User Manual, section 4.5:\\[3ex]
  \includegraphics[width=\textwidth]{catch_conditioned}
\end{frame}

% ______________________________________________________________________________

\begin{frame}{Estimation Uncertainty}
  \vspace{1ex}
  \centering\includegraphics[height=0.7\textheight]{estimation_uncertainty}
\end{frame}

% ______________________________________________________________________________

\begin{frame}\Large
  \centering\darkgreen\bf Issues in 2023 Assessment
\end{frame}

% ______________________________________________________________________________

\begin{frame}{Convergence and Jittering}
  \vspace{1ex}
  \centering\includegraphics[height=0.65\textheight]{jittering}
\end{frame}

% ______________________________________________________________________________

\begin{frame}{Narrow Confidence Intervals}
  \begin{itemize}
    \item Incorporating estimation uncertainty had no real effect on the overall
    uncertainty.\\[0.5ex]
    For example, the confidence interval for
    $SB_\mathrm{recent}/SB_{F=0}$:\\[1ex]
    \quad $0.42\!-\!0.52$ without estimation uncertainty\\[0.5ex]
    \quad $0.42\!-\!0.52$ including estimation uncertainty\\[6ex]
    \item Also, the confidence interval around the estimated M is quite narrow:
    $0.11\!-\!0.13$\\[1ex]
  \end{itemize}
  \centering\includegraphics[width=0.89\textwidth]{natmort_confint_2023}
  \vspace{2ex}
\end{frame}

% ______________________________________________________________________________

\begin{frame}{Likelihood Conflict Between Length and Weight Comps}
  \vspace{1ex}
  \centering\includegraphics[height=0.75\textheight]{likelihood_profile}
\end{frame}

% ______________________________________________________________________________

\begin{frame}\Large
  \centering\darkgreen\bf New Issues
\end{frame}

% ______________________________________________________________________________

\begin{frame}{Uncertainty About Indonesian Catches}
  \vspace{1ex}
  \centering\includegraphics[height=0.75\textheight]{indonesia_1}
\end{frame}

% ______________________________________________________________________________

\begin{frame}{Uncertainty About Indonesian Catches}
  \vspace{1ex}
  \centering\includegraphics[height=0.75\textheight]{indonesia_2}
\end{frame}

% ______________________________________________________________________________

\begin{frame}{Uncertainty About Indonesian Catches}
  \vspace{1ex}
  \centering\includegraphics[height=0.75\textheight]{indonesia_2}
\end{frame}

% ______________________________________________________________________________

\begin{frame}{Uncertainty About Indonesian Catches}
  \vspace{1ex}
  \centering\includegraphics[height=0.75\textheight]{indonesia_3}
\end{frame}

% ______________________________________________________________________________

\begin{frame}\Large
  \centering\darkgreen\bf New Features in MULTIFAN-CL
\end{frame}

% ______________________________________________________________________________

\begin{frame}{Length-based selectivity}
  Email from Nick Davies yesterday:\\[3ex]
  \begin{quotation}\noindent
    Yes, the length-based selectivity feature as used for SKJ2025 can be applied
    to any stock assessment. Note that it differs fundamentally from that of
    SS3, in that it conserves the separability assumption for time-invariant
    selectivity-at-age for the fishing mortality calculations.
  \end{quotation}
\end{frame}

% ______________________________________________________________________________

\begin{frame}\Large
  \centering\darkgreen\bf Things to Report to PAW
\end{frame}

% ______________________________________________________________________________

\begin{frame}{Things to Report to PAW}
  \vspace{2ex}
  \begin{itemize}
    \item[] {\bf\darkblue ~}
    \comment{~\\[1ex]
      \h{38.5ex} ~}\\[4ex]
    \item[] {\bf\darkblue Issues in 2023 Assessment}
    \comment{convergence/jittering, narrow confidence intervals,\\[1ex]
      \h{32.2ex} likelihood conflict between length and weight comps}\\[4ex]
    \item[] {\bf\darkblue New Issues}
    \comment{uncertainty about Indonesian catches}\\[4ex]
    \item[] {\bf\darkblue New Features in MULTIFAN-CL}
    \comment{length-based selectivity}\\[4ex]
    \item[] {\bf\darkblue ~}
    \comment{~}\\[4ex]
  \end{itemize}
\end{frame}

% ______________________________________________________________________________

\begin{frame}
  \vspace{4ex}
  These slides:\\[2ex]
  \begin{center}
    \blue\url{https://github.com/PacificCommunity/ofp-sam-yft-2026-assessment}
  \end{center}
\end{frame}

\end{document}
